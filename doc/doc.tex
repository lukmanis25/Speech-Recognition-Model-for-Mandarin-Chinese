%%%%%%%%%%%%%%%%%%%%%%%%%%%%%%%%%%%%%%%%%
% Journal Article
% LaTeX Template
% Version 2.0 (February 7, 2023)
%
% This template originates from:
% https://www.LaTeXTemplates.com
%
% Author:
% Vel (vel@latextemplates.com)
%
% License:
% CC BY-NC-SA 4.0 (https://creativecommons.org/licenses/by-nc-sa/4.0/)
%
% NOTE: The bibliography needs to be compiled using the biber engine.
%
%%%%%%%%%%%%%%%%%%%%%%%%%%%%%%%%%%%%%%%%%

%----------------------------------------------------------------------------------------
%	PACKAGES AND OTHER DOCUMENT CONFIGURATIONS
%----------------------------------------------------------------------------------------

\documentclass[
	a4paper, % Paper size, use either a4paper or letterpaper
	10pt, % Default font size, can also use 11pt or 12pt, although this is not recommended
	unnumberedsections, % Comment to enable section numbering
	twoside, % Two side traditional mode where headers and footers change between odd and even pages, comment this option to make them fixed
]{LTJournalArticle}

\addbibresource{sample.bib} % BibLaTeX bibliography file

% \usepackage{xcolor}

\runninghead{Chinese Pronunciation Evaluation} % A shortened article title to appear in the running head, leave this command empty for no running head

\footertext{\textit{Politechnika Gdańska} (2024)} % Text to appear in the footer, leave this command empty for no footer text

\setcounter{page}{1} % The page number of the first page, set this to a higher number if the article is to be part of an issue or larger work

%----------------------------------------------------------------------------------------
%	TITLE SECTION
%----------------------------------------------------------------------------------------

\title{A Speech Recognition Model for Mandarin Chinese Pronunciation Evaluation} % Article title, use manual lines breaks (\\) to beautify the layout

% Authors are listed in a comma-separated list with superscript numbers indicating affiliations
% \thanks{} is used for any text that should be placed in a footnote on the first page, such as the corresponding author's email, journal acceptance dates, a copyright/license notice, keywords, etc
\author{%
Jakub Kiliańczyk, % 184301 (kierownik zespołu)
Jakub Kwiatkowski, % 184348 
Anna Strzelecka, \\ % 201986
Dawid Migowski, % 184819
Łukasz Smoliński % 184306
% \\ \textit{\color{red} ? Opiekun: Adam Przybyłek }
}


% Affiliations are output in the \date{} command
\date{\footnotesize{Politechnika Gdańska}}

% Full-width abstract
\renewcommand{\maketitlehookd}{%
	\begin{abstract}
		\noindent Nasz projekt jest bardzo fajny i ciekawy. W dzisiejszych czasach AI jest do wszystkiego, dlatego zdecydowaliśmy się zastosować sieci neuronowe do problemu sprawdzania poprawności wymowy i intonacji Mandaryńskiego.
	\end{abstract}
}

%----------------------------------------------------------------------------------------

\begin{document}

\maketitle % Output the title section

%----------------------------------------------------------------------------------------
%	ARTICLE CONTENTS
%----------------------------------------------------------------------------------------

\section{Introduction}

Na podstawie naszych źródeł XYZ zdecydowaliśmy się zrealizować zadanie za pomocą sieci konwolucyjnych (sPlOtOwYcH)
oraz modelu Transformer. % Porównaliśmy wyniki wytrenowanych modeli i wyszło nam to to i siamto.
Wykorzystaliśmy dane ze zbioru [AISHELL-3] oraz własne dane zebrane drogą eksperymentu, który polegał na bla bla bla.
Aby lepiej wykorzystać potencjał danych uczących zastosowaliśmy cośtam i jeszcze cośtam.
Tak btw to nw czy to piszemy po polsku czy angielsku, może na początek niech zostanie po pl i się potem przetłumaczy.

%------------------------------------------------

\section{Related Work}

Przeprowadziliśmy SLR i znaleźliśmy kilka rozwiązań głównie pozwalających na jedynie połowiczną realizację założeń tego projektu.
Tj. albo sam goły ASR sprowadzający się do sprawdzania wymowy, albo rozpoznawanie tonu niezależnie od artykulacji.
BTW Jakby jakieś elementy typu podsekcje, listy, równania itd. były potrzebne to polecam kopiować z template.tex.

%------------------------------------------------

\section{Approach}

Tutaj opisujemy nasze podejścia dużo bardziej szczegółowo, każdy sobie rzepkę skrobie

\subsection{Model per słowo}

Osobne modele sieci konwolucyjnych do każdego słowa.
Można napisać jak robiliśmy na początku i co potem zmienialiśmy żeby lepiej zadziałało.

\subsection{One model to rule them all}

Sieć konwolucyjna wykorzystująca embedding (pl: WSAD xD) żeby działać na wszystkich słowach.

\subsection{Model do tonów by Dawid}

Opis jak model Dawida działa.

\subsection{Transformer}

Opis jak model Kuby Ki działa

%------------------------------------------------

\section{Results}

Wyniki były takie i śmakie. Porównujemy wyniki różnych podejść i modeli, może z różnymi hiperparametrami żeby się bardziej rozpisać.

%------------------------------------------------

\section{Discussion}

No ogólnie to widać że jest potencjał, przyszłe pokolenia badaczy mogą podnieść temat tu gdzie my go odkładamy i pewnie wyniki będą coraz lepsze.

%------------------------------------------------

\section{Conclusion}

To i to wyszło, co innego nie wyszło z takich a takich przyczyn.

\textbf{\color{red} cytowania dodajemy w sample.bib (wyżej jest to dołączane znaczikiem addbibresource)}
\textbf{\color{red} żeby się wyświetliły na dole trzeba gdzieś w tekście umieścić autocite, jak np. tutaj \autocite{TODO:2024}}

%----------------------------------------------------------------------------------------
%	 REFERENCES
%----------------------------------------------------------------------------------------

\printbibliography % Output the bibliography

%----------------------------------------------------------------------------------------

\end{document}
